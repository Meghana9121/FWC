\documentclass[journal,12pt,twocolumn]{IEEEtran}
%
\usepackage{setspace}
\usepackage{textcomp}
\usepackage{gensymb}
\usepackage{xcolor}
\usepackage{caption}
%\usepackage{subcaption}
%\doublespacing
\singlespacing

\usepackage{graphicx}
%\usepackage{amssymb}
%\usepackage{relsize}
\usepackage[cmex10]{amsmath}
\usepackage{mathtools}
%\usepackage{amsthm}
%\interdisplaylinepenalty=2500
%\savesymbol{iint}
%\usepackage{txfonts}
%\restoresymbol{TXF}{iint}
%\usepackage{wasysym}
\usepackage{amsthm}
\usepackage{mathrsfs}
\usepackage{txfonts}
\usepackage{stfloats}
\usepackage{cite}
\usepackage{cases}
\usepackage{subfig}
%\usepackage{xtab}
\usepackage{longtable}
\usepackage{multirow}
%\usepackage{algorithm}
%\usepackage{algpseudocode}
\usepackage{enumitem}
\usepackage{mathtools}
%\usepackage{eenrc}
%\usepackage[framemethod=tikz]{mdframed}
\usepackage{hyperref}
\usepackage{listings}
    \usepackage[latin1]{inputenc}                                 %%
    \usepackage{color}                                            %%
    \usepackage{array}                                            %%
    \usepackage{longtable}                                        %%
    \usepackage{calc}                                             %%
    \usepackage{multirow}                                         %%
    \usepackage{hhline}                                           %%
    \usepackage{ifthen}                                           %%
  %optionally (for landscape tables embedded in another document): %%
    \usepackage{lscape}     
\usepackage{tikz}
\usepackage{circuitikz}
\usepackage{karnaugh-map}
\usepackage{pgf}

\usepackage{url}
\def\UrlBreaks{\do\/\do-}



%\usepackage{stmaryrd}


%\usepackage{wasysym}
%\newcounter{MYtempeqncnt}
\DeclareMathOperator*{\Res}{Res}
%\renewcommand{\baselinestretch}{2}
\renewcommand\thesection{\arabic{section}}
\renewcommand\thesubsection{\thesection.\arabic{subsection}}
\renewcommand\thesubsubsection{\thesubsection.\arabic{subsubsection}}

\renewcommand\thesectiondis{\arabic{section}}
\renewcommand\thesubsectiondis{\thesectiondis.\arabic{subsection}}
\renewcommand\thesubsubsectiondis{\thesubsectiondis.\arabic{subsubsection}}



%\surroundwithmdframed[width=\columnwidth]{lstlisting}
\def\inputGnumericTable{}                                 %%
\lstset{
%language=C,
frame=single, 
breaklines=true,
columns=fullflexible
}
 

\begin{document}
%

\theoremstyle{definition}
\newtheorem{theorem}{Theorem}[section]
\newtheorem{problem}{Problem}
\newtheorem{proposition}{Proposition}[section]
\newtheorem{lemma}{Lemma}[section]
\newtheorem{corollary}[theorem]{Corollary}
\newtheorem{example}{Example}[section]
\newtheorem{definition}{Definition}[section]
%\newtheorem{algorithm}{Algorithm}[section]
%\newtheorem{cor}{Corollary}
\newcommand{\BEQA}{\begin{eqnarray}}
\newcommand{\EEQA}{\end{eqnarray}}
\newcommand{\define}{\stackrel{\triangle}{=}}
\vspace{3cm}
\title{ 
Staircase Switch through Arm
}

\author{V.MEGHANA}


\maketitle
\tableofcontents
\bigskip
%
%\newpage
\section{Abstract}

This manual shows the implementation of the Staircase switch using XNOR GATE .
\section{\textbf{Components}}
\input{components}

%\chapter{Tables}
\begin{table}[ht]
\centering
\begin{tabular}{|l|c|c|r|}
\hline
Input & Gnd & Vcc & led\\
\hline
Arm & 4 & 5 & 22\\
\hline
\end{tabular}
\caption{}
\label{tab:first table}
\end{table}
\section{Description}

1. Connect the circuit as per Table I\\
2. Vary the 4 and 5 of ESP32 and observe the output 
accordingly in the LED.

\section{Procedure}
\raggedright 1.After executing the following code using make, a binary file is generated with .bin extension in the output directory. \vspace{2mm} \\ 
\raggedright 2.Now from the termux, using scp protocol, send the generated bin file to the laptop. \\ \vspace{2mm}
\raggedright 3.There we are supposed to flash the .bin file into the ARM through the terminal.\\ \vspace{2mm}
\raggedright 4.After flashing, reset the Vaman board.\\ \vspace{2mm}
\raggedright 5.Make connections between the LED and ARM board using jumper wires. \\ \vspace{2mm}
\raggedright 6.Now check the output with reference to the truth table present above.


%\begin{figure}
 %   \centering
  % \caption{7447}
   % \label{fig:circuit}
%\end{figure}
\begin{figure}
    \centering
    \includegraphics[width=2in]{xnor.jpeg}
    \caption{Switches OFF-ON}
    \label{fig:circuit}
\end{figure}

\begin{figure}
    \centering
    \includegraphics[width=2in]{Staircase.jpeg}
    \caption{Staircase Switch when both are on}
    \label{fig:circuit}
\end{figure}


\begin{table}[ht]
\centering
%\resizebox{\columnwidth}{!}{%
\begin{tabular}{|l|l|l|l|}
\hline
 & \textbf{W} & \textbf{X} & \textbf{Z} \\ \hline
\textbf{0} & 0 & 0 & 1 \\ \hline
\textbf{1} & 0 & 1 & 0 \\ \hline
\textbf{2} & 1 & 0 & 0 \\ \hline
\textbf{3} & 1 & 1 & 1 \\ \hline
\end{tabular}%
%}
\caption{Truth table}
\label{Truth table}
\end{table}
\section{Code}

\textbf{Observe the circuit by executing the link provided below.}\\
\begin{center}
\fbox{\parbox{8.5cm}{\url{https://github.com/Meghana9121/FWC/Arm}}}
\end{center}
%https://github.com/Meghana9121/FWC/Arm
\end{document}
